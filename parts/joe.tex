\documentclass[../article.tex]{subfiles}

\begin{document}

\section{The Discrete Fourier Transform}
Our purpose in this section is to give the reader an understanding of the theoretical framework which gives rise to the Discrete Fourier transform.

\subsection{The Signal Interpretation of the Discrete Fourier Transform}

Let $f(t)$ be a periodic function. Then the discrete Fourier transform of $f(t)$, called $\mathcal{F}\{f(t)\}$, is the value of the amount of frequency given by $f(t)$. Consider a set of samples $S = \{0,1,2,...,T-1\}$. Then the discrete Fourier transform of $f(S)$ is

\[
F(n) = \sum_{k=0}^{T-1} f(k) e^{-2 \pi ikn/T}
\]

where $n \in \mathbb{Z}$.

\subsection{The Group $\mathbb{Z}(N)$}

As it turns out, the group $\mathbb{Z}(N)$ plays an important part in the theory of the discrete Fourier transform, and acts as a weight for storing information on the unit circle for any given periodic function.

\begin{definition}
Let $\mathbb{Z}(N)$ be the set of all N-th roots of unity. Then,
 \[Z(N) = \{z \in C : z^N = 1 \}\]
\end{definition}
\begin{theorem}
For all $N \in \mathbb{Z}$, we have
\[\mathbb{Z}(N)= \left\{1, e^{(2 \pi i)/N}, e^{(4 \pi i)/N}, \ldots, e^{(2(N-1) \pi i)/N}\right\}\]
\end{theorem}

\begin{proof}
    Let $z = re^{i\theta} \in \mathbb{Z}(N)$. Then:
    \[
        |(re^{i\theta})^N| = |r^N| = 1
    \]
    So we know that $z = e^{i\theta}$. Next for any 
    $0 \leq k < N$ we know that:
    \[
        (e^{2\pi ik/N})^N = e^{2\pi ik} = 1
    \]
    Thus all of these are roots of unity. Moreover it is 
    easy to see that for $k' < 0$ or $k' \geq N$ that there
    exists some $0 \leq k < N$ such that $k' \equiv k \mod N$.
    Additionally if $\theta \neq 2\pi if/N$, for some $f \in 
    \mathbb{Z}$ then
    \[
        (e^{i\theta})^N = e^{iN\theta} \neq e^{2\pi if} = 1
    \]
    Thus not only are all of these numbers roots of unity,
    they are the only roots of unity.
\end{proof}

We can also show that $\mathbb{Z}(N)$ is a finite abelian group and, indeed, forms a basis of a vector space of complex-valued functions on the unit circle.

\begin{theorem}
$\mathbb{Z}(N)$ is a finite abelian group.
\end{theorem}

\begin{proof}
    We already showed that $\mathbb{Z}(N)$ has $N < \infty$
    elements thus is is finite. More over if $z_1,z_2 \in
    \mathbb{Z}(N)$ we have:
    \[
        (z_1z_2^{-1})^N = \frac{z_1^N}{z_2^N} = \frac{1}{1} = 1
    \]
    Thus $z_1z_2^{-1} \in \mathbb{Z}(N)$ and via the subgroup
    test $\mathbb{Z}(N)$ is a subgroup of $\mathbb{C}$.
\end{proof}

Furthermore it is the case that $\mathbb{Z}(N)$ is isomorphic
to $\mathbb{Z}/N\mathbb{Z}$

\begin{theorem}
$\mathbb{Z}(N)$ is isomorphic to $\mathbb{Z}/N\mathbb{Z}$
\end{theorem}

\begin{proof}
    We note that for any element $z \in \mathbb{Z}(N)$
    that:
    \[
        z = e^{2\pi in/N} = (e^{2\pi/N})^n
    \]
    Thus $\mathbb{Z}(N)$ is a cyclic group of order $N$ 
    which shows it is isomorphic to $\mathbb{Z}/N\mathbb{Z}$
\end{proof}
Now that we have proven $\mathbb{Z}(N)$ is a finite abelian group  isomorphic to $\mathbb{Z} /N\mathbb{Z}(N)$, we can think of the discrete Fourier transform as a vector space over $\mathbb{Z}(N)$ with the Hermitian inner product


\[
  (F,G) = \sum_{n=0}^{N-1} F(k) \overline{G(k)}
\]
and norm

\[
  ||F||^2 = \sum_{k=0}^{N-1} |F(k)|^2
\]


\begin{theorem}
If $F$ is a function on $\mathbb{Z}(N)$, then
\[
F(k) = \sum_{n=0}^{N-1} a_n e^{2 \pi nk/N}
\]
where
\[
a_n = \frac{1}{N} \sum_{k=0}{N-1} F(k) e^{-2 \pi ikn/N}.
\]
\end{theorem}
\begin{proof}
Let $\zeta^{lk} = e^{2 \pi i lk/N}$, where $k,l = 0,1,...N-1$. Then, the set
\[
V = \left\{ \frac{1}{\sqrt{N}}, \frac{1}{\sqrt{N}}e^{2 \pi i k/N}, \ldots, \frac{1}{\sqrt{N}}e^{2 \pi i k(N-1)/N} \right\}
\]
forms an orthogonal basis for $\mathbb{Z}(N)$. So,
\[
F(k) = \sum_{n=0}^{N-1} \Big(F(k), \frac{1}{\sqrt{N}}\zeta^{nk} \Big) \frac{1}{\sqrt{N}}\zeta^{nk}
\]
where $\big(F,G \big)$ represents the Hermitian product of $F$ and $G$. By substitution we obtain
\begin{align*}
F(k) &= \sum_{n=0}^{N-1} \Bigg(\sum_{k=0}^{N-1} F(k) \frac{1}{\sqrt{N}} \zeta^{-lk} \Bigg) \frac{1}{\sqrt{N}} \zeta^{lk} \\
&= \sum_{n=0}^{N-1} \Bigg(\frac{1}{N} \sum_{k=0}^{N-1} F(k) \zeta^{-lk} \Bigg) \zeta^{lk} \\
&= \sum_{n=0}^{N-1} a_n e^{2 \pi nk/N}
\end{align*}
where
\[
a_n = \frac{1}{N} \sum_{k=0}^{N-1} F(k) e^{-2 \pi ikn/N}
\]
and $e^{2 \pi i/N} = \zeta$.
\end{proof}

With the discrete Fourier transform defined, it is now possible to define the Fast Fourier Transform, which is the subject of the next section.

\end{document}
