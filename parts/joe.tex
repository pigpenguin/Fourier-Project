\documentclass[../article.tex]{subfiles}

\begin{document}

\section{The DFT and other nonsense}
Our purpose in this section is to give the reader an understanding of the theoretical framework which gives rise to the Discrete Fourier transform and then give some details on other related concepts.

\subsection{The Signal Interpretation of the Discrete Fourier Transform}

Let $f(t)$ be a periodic function. Then the discrete Fourier transform of $f(t)$, called $\mathcal{F}\{f(t)\}$, is the value of the amount of frequency given by $f(t)$. Consider a set of samples $S = \{0,1,2,...,T-1\}$. Then the discrete Fourier transform of $f(S)$ is

\[
F(n) = \sum_{k=0}^{T-1} f(k) e^{-2 \pi ikn/T}
\]

where $n \in \mathbb{Z}$.

\subsection{The Group $\mathbb{Z}(N)$}

As it turns out, the group $\mathbb{Z}(N)$ plays an important part in the theory of the discrete Fourier transform, and acts as a weight for storing information on the unit circle for any given periodic function.

\begin{definition}
Let $\mathbb{Z}(N)$ be the set of all N-th roots of unity. Then,
 \[Z(N) = \{z \in C : z^N = 1 \}\]
\end{definition}
\begin{theorem}
For all $N \in \mathbb{Z}$, we have that
\[\mathbb{Z}(N)= \left\{1, e^{(2 \pi i)/N}, e^{(4 \pi i)/N}, \ldots, e^{(2(N-1) \pi i)/N}\right\}\]
\end{theorem}

\begin{proof}
Let $z$ $\in \mathbb{Z}(N)$. Then, $z^N = (r \cdot e^{i \theta})^N = 1$. Therefore, $\mid r^N \mid \mid e^{i N \theta} \mid = 1$. Hence, $r = e^{i N \theta} = 1$. Therefore, $N\theta = 2k\pi$, where $k \in \mathbb{Z}$. Let $\zeta$ = $e^{2 \pi i /(N-1)}$. Then,
\begin{align*}
	(\zeta^0)^N &= e^0 = 1 \\
	(\zeta^1)^N &= e^{2 \pi i} = 1 \\
	(\zeta^2)^N &= e^{4 \pi i} = 1 \\
	{} &\vdotswithin{=}\\
	(\zeta^{N-1})^N &= (e^{2 \pi i(N-1)/N})^N \\
	{} &= (e^{2 \pi i}e^{-2 \pi i/N})^N \\
	{} &= 1 \cdot e^{-2 \pi i} \\
	{} &= 1
\end{align*}
Therefore, $\zeta^k \in \mathbb{Z}(N)$ for all $k$ such that $0 \leq k \leq N-1$. Notice that the set $Q = \{0, 1, 2, \ldots, N-1\}$ is a residue system modulo $N$, and that the set $\zeta^Q = \left\{1, e^{2 \pi i/N}, e^{4 \pi i/N}, \ldots, e^{2(N-1) \pi i/N}\right\} \subseteq \mathbb{Z}(N) $.
Now, for every $n \in \mathbb{Z}$, there exists an integer $m \in Q$ such that $n \equiv m \mod N$, since $Q$ is a complete residue system modulo $N$. When $k \geq N-1$, we have that $k \equiv l \mod N$ for some $l \in Q$. Therefore, $\mathbb{Z}(N) \subseteq Q$, and since we know $Q \subseteq Z(N)$, we have $Q = Z(N)$.
\end{proof}

We can also show that $\mathbb{Z}(N)$ is a finite abelian group and, indeed, forms a basis of a vector space of complex-valued functions on the unit circle.

\begin{theorem}
$\mathbb{Z}(N)$ is a finite abelian group.
\end{theorem}

\begin{proof}
That $| \mathbb{Z}(N)| < \infty$ is self-edvident. Thus, $\mathbb{Z}(N)$ is a finite group. Letting $z,w \in \mathbb{Z}(N)$, we find that $z \cdot w = e^{2 \pi i (k+l)/N}$ for some $k,l \in \mathbb{Z}$. Then, by the complete residue theorem of number theory, there exists an $m \in \{1, 2, 3, ..., N-1\}$ such that $(k+l) \equiv m \mod N$. Hence, $e^{2 \pi i (k+l)/N} = e^{2 \pi i m/N} \in \mathbb{Z}(N)$. Therefore, $\mathbb{Z}(N)$ is closed under complex multiplication. Let $z,w \in \mathbb{Z}(N)$. Then $z \cdot w = e^{2 \pi i k/N} \cdot e^{2 \pi i l/N} = e^{2 \pi i (k+l)/N} = e^{2 \pi i (l+k)/N} = e^{2 \pi i l/N} \cdot e^{2 \pi i k/N} = w \cdot z$. Therefore $(\mathbb{Z}(N), \cdot)$ is commutative. Next, let $z,w,s \in \mathbb{Z}(N)$. Then, $(z \cdot w) \cdot s = (e^{2 \pi i k/N} \cdot e^{2 \pi i l/N}) \cdot e^{2 \pi i t/N} = e^{2 \pi i ((k+l)+t)/N} = e^{2 \pi i (k+(l+t))/N} = e^{2 \pi i k/N} \cdot (e^{2 \pi i l/N} \cdot e^{2 \pi i t/N}) = z \cdot (w \cdot s)$. So $(\mathbb{Z}(N), \cdot)$ is associative. The identity element is $1 \in \mathbb{Z}(N)$, since for all $z \in \mathbb{Z}(N)$ we have $z \cdot 1 = z$. If $z \in \mathbb{Z}(N)$, then $(1/z)^N = 1/ z^N = 1$. Therefore, $1/z \in \mathbb{Z}(N)$ and the inverse of any $z \in \mathbb{Z}(N)$ is $1/z$. Therefore, $\mathbb{Z}(N)$ is an abelian group.
\end{proof}

Furthermore, we can show that $\mathbb{Z}/ N\mathbb{Z}$, the set of all equivalence classes of integers modulo $N$, is an abelian group.

\begin{definition}
For all $x \in \mathbb{Z}$, let $R_N (x) = \{y \in \mathbb{Z} : y \equiv x   \mod   N\}$, where $N \in \mathbb{Z}$
\end{definition}
\begin{theorem}
For all $x,y \in \mathbb{Z}$, $R_N (x) + R_N (y) = R_N (x+y).$
\end{theorem}
\begin{proof}
Let $\alpha \in R_N (x)$ and $\beta \in R_N (y)$. Then, $\alpha = x +kN$ for some $k \in \mathbb{Z}$ and $\beta = y + lN$ for some $l \in \mathbb{Z}$. Then, $(x+y)+(k+l)N \in R_N (x+y)$. Conversely, let $x+y \in R_N (x+y)$. Then, $x = \alpha -kN$ and $y = \beta -lN$. Therefore, $(x+y) = (\alpha + \beta) -(k+l)N$. Therefore, $(x+y) \equiv \alpha +\beta) \mod N$. Therefore, $x \equiv \alpha \mod N$ and $y \equiv \beta \mod N$. Therefore, $x+y \in R_N (x) + R_N (y)$.
\end{proof}

Now that we have proven $\mathbb{Z}(N)$ is a finite abelian group and homomorphic to the group $\mathbb{Z} /N\mathbb{Z}(N)$, we can think of the discrete Fourier transform as a vector space over $\mathbb{Z}(N)$ with the Hermitian inner product

\[
  (F,G) = \sum_{n=0}^{N-1} F(k) \overline{G(k)}
\]
and norm

\[
  ||F||^2 = \sum_{k=0}^{N-1} |F(k)|^2
\]

\begin{theorem}
If $F$ is a function on $\mathbb{Z}(N)$, then
\[
F(k) = \sum_{n=0}^{N-1} a_n e^{2 \pi nk/N}
\]
where
\[
a_n = \frac{1}{N} \sum_{k=0}{N-1} F(k) e^{-2 \pi ikn/N}.
\]
\end{theorem}
\begin{proof}
Let $\zeta^{lk} = e^{2 \pi i lk/N}$, where $k,l = 0,1,...N-1$. Then, the set
\[
V = \left\{ \frac{1}{\sqrt{N}}, \frac{1}{\sqrt{N}}e^{2 \pi i k/N}, \ldots, \frac{1}{\sqrt{N}}e^{2 \pi i k(N-1)/N} \right\}
\]
forms an orthogonal basis for $\mathbb{Z}(N)$. So,
\[
F(k) = \sum_{n=0}^{N-1} \Big(F(k), \frac{1}{\sqrt{N}}\zeta^{nk} \Big) \frac{1}{\sqrt{N}}\zeta^{nk}
\]
where $\big(F,G \big)$ represents the Hermitian product of $F$ and $G$. By substitution we obtain
\begin{align*}
F(k) &= \sum_{n=0}^{N-1} \Bigg(\sum_{k=0}^{N-1} F(k) \frac{1}{\sqrt{N}} \zeta^{-lk} \Bigg) \frac{1}{\sqrt{N}} \zeta^{lk} \\
&= \sum_{n=0}^{N-1} \Bigg(\frac{1}{N} \sum_{k=0}^{N-1} F(k) \zeta^{-lk} \Bigg) \zeta^{lk} \\
&= \sum_{n=0}^{N-1} a_n e^{2 \pi nk/N}
\end{align*}
where
\[
a_n = \frac{1}{N} \sum_{k=0}^{N-1} F(k) e^{-2 \pi ikn/N}
\]
and $e^{2 \pi i/N} = \zeta$.
\end{proof}

With the discrete Fourier transform defined, it is now possible to define the Fast Fourier Transform, which is the subject of the next section.

\end{document}
