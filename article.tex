\documentclass[11pt]{report}
%Allow subfiles for more collaboration
  \usepackage{subfiles}

%Make latex do math more good
  \usepackage{mathtools}
  \usepackage{amssymb}
  \usepackage{amsthm}

%More floats
  \usepackage{morefloats}

%Script font for tom
  \usepackage{mathrsfs}

% Handling graphics, telling latex where to find em
  \usepackage{graphicx}
  \graphicspath{{img/}{../img/}}

% Bibliography Resources
  \usepackage{biblatex}
  \addbibresource{article.bib}

% Removing the "Chapter" from chapter headings
  \usepackage{titlesec}
  \titleformat{\chapter}{\normalfont\huge}{\thechapter.}{20pt}{\huge\it}

% Fixing latex's terrible default tables
  \usepackage{booktabs}
  \usepackage[table]{xcolor}
  \setlength{\tabcolsep}{7pt}
  \renewcommand{\arraystretch}{1.7}
  \renewcommand{\lightrulewidth}{.025em}
  \arrayrulecolor{gray}

%We want spaces between paragraphs, not indents
  \setlength{\parindent}{0em}
  \setlength{\parskip}{1em}

%Custom operators we want
  \DeclareMathOperator{\sinc}{sinc}

%Custom theorem enviroments
  \newtheorem{theorem}{Theorem}
  \newtheorem{definition}{Definition}
  \newtheorem{corollary}{Corollary}
  \newtheorem{references}{References}


\title{Discrete Fourier Transform and its Applications}

\author{Tom Hsieh\\
        Evar Jimenez\\
        Joseph Knox\\
        Jocelyn Park \\
        Ethan Rooke }

\begin{document}
\maketitle

\tableofcontents

\chapter{Theory}
  The goal of this paper is to discuss real world applications
  of Fourier analysis. We will be looking at applications to
  signal processing and sound analysis, image processing, and
  finding paterns in time varying data sets. Before we can
  dive into these subjects we seek to build up a theoretical
  backbone to provide the math needed. To this end this
  section will describe the math behind the Discrete Fourier Transform
  which is how we analyze this data and the Fast Fourier Transform
  which is how we get computers to calculate these objects rapidly.
  \subfile{./parts/joe}
  \newpage
  \subfile{./parts/tom}
  \newpage
\chapter{Applications}
  Now that we have developed the tools needed we will look at
  why we wanted them in the first place. We will look at three
  applications and discuss how you can use Fourier Analysis to
  maniuplate and understand the data. We will be looking at how
  to discretely record and reproduce sound, manipulate images,
  and find patterns in various data sets. The DFT is the tool
  which allows us to do this while the FFT gives us a feasible
  way to have computers compute these quickly.
  \subfile{./parts/ethan}
  \newpage
  \subfile{./parts/jocelyn}
  \newpage
  \subfile{./parts/evar}

  \nocite{*} %print our entire bibliography
  \printbibliography
\end{document}
