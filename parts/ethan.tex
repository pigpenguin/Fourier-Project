\documentclass [../article.tex]{subfiles}

\begin{document}
  \section{Signal Processing}
  One major area where Fourier Series are used is signal
  processing. They allow you to dissect a signal and analyze
  it's component parts among other things. The particular use
  we will be looking at this section is how to record and reproduce
  a band limited signal digitally with no loss of information.
  This is of particular interest to the music industry as they
  need a way to record, store, and reproduce sound on computers.

  \subsection{Recording}
  The primary issue is that sound and other signals are continuous
  while digital computers are inherently discrete. Thus we need
  to sample at some high enough frequency so as to not lose any
  information. Intuitively this feels like it would be impossible,
  surely if you sample finitely many points something could happen
  in between them that they wouldn't register. This turns out to not
  be the case for band limited signals as we will soon see.

  Let $x(t)$ be some signal and $X(\omega)$ be it's
  spectrum, which is a map from a frequency to it's
  strength in the signal. It is apparent if you know $x(t)$
  you can compute $X(\omega)$ and if you know $X(\omega)$,
  you can compute $x(t)$. Thus $x$ and $X$ uniquely determine
  each other. We will assume that the signal is band limited, i.e. no
  frequencies above or below a certain level. In the case of
  sound even if your signal is not band limited our hearing is,
  so you can just get rid of frequencies outside of our range.
  We will also assume your signal is nice enough to have a
  Fourier series representation naturally produced signals such
  as sound are this nice. The following argument is due to
  \cite{shannon1949}.

  The relation between $x(t)$ and $X(\omega)$ can be given by:
  \[x(t) = \frac{1}{2\pi}\int_{-\infty}^\infty
  X(\omega)e^{it\omega}\,d\omega\]
  However as we are band limited we know that if
  $|\omega| > \frac{B}{2\pi}$ for some $B$ that
  $X(\omega) = 0$ thus our integral becomes
  \[x(t) = \frac{1}{2\pi}\int_{-2\pi B}^{2\pi B}
  X(\omega)e^{it\omega}\,d\omega\]
  Next we choose some time interval to sample our signal at.
  For instance consider sampling at the intervals $\frac{n}{2B}$
  where $n \in \mathbb{Z}$. This yields:
  \[x\left(\frac{n}{2B}\right) =
  \int_{-2\pi B}^{2\pi B}X(\omega)e^{\frac{itn}{2B}}\,d\omega\]
  This, however, is just the $n^{\text{th}}$ Fourier coefficient
  of $X(\omega)$ so we have:
  \[x\left(\frac{n}{2B}\right) = \hat{X}(n)\]
  Thus if we sample our signal at regular intervals of
  $n/2B$ we will get our Fourier series for $X(n)$ which
  uniquely determines $X(n)$ which uniquely determines $x(t)$.

  \subsection{Recreation}
  Now let's assume you have these data points at $t=\frac{n}{2B}$
  how would you go about recreating the sound? To do this you use
  the Whittaker-Shannon interpolation theorem which states that:
  \[f(t) = \sum_{k=-\infty}^\infty f\left(\frac{k}{2B}\right)\sinc(2Bt-k)\]
  Where we have defined:
  \[\sinc(x) = \begin{cases}
                  \frac{\sin(\pi x)}{\pi x} & x \neq 0\\
                  0 & x = 0
                \end{cases} \]
  To prove this we use the argument from \cite{vrscay_2008}.
  \begin{proof}
    We start by writing out the Fourier Series for $X$
    \[X(\omega)=\sum_{k=\infty}^\infty\hat{X}(k)e^{i\pi k\omega/\pi}\]
    We then turn our attention to $\hat{X}(k)$ and get the following
    expression:
    \begin{align*}
      \hat{X}(k) &= \frac{1}{{4\pi B}}\int_{-{2\pi B}}^{2\pi B} X(\omega)
      e^{-i\pi k \omega/{2\pi B}}\,d\omega\\
      {} &= \frac{1}{{4\pi B}}\int_{-\infty}^\infty X(\omega)
      e^{-i\pi k \omega/{2\pi B}}\,d\omega\\
      {} &= \frac{\sqrt{2\pi}}{{4\pi B}}\frac{1}{\sqrt{2\pi}}
      \int_{-\infty}^\infty X(\omega) e^{-i\pi k \omega/{2\pi B}}\,d\omega\\
      {} &= \frac{\sqrt{2\pi}}{{4\pi B}}x\left(\frac{k}{2B}\right)
    \end{align*}
    We take this expression for $\hat{X}$ and substitute back
    into our original expression to obtain:
    \[ X(\omega) = \frac{\sqrt{2\pi}}{{4\pi B}}\sum_{k=-\infty}^\infty
    x\left(\frac{k}{2B}\right)e^{-i\pi k\omega/{2\pi B}}\]
    We now use this representation of $X(\omega)$ to calculate
    $x(t)$ using the Inverse Fourier Transform:
    \begin{align*}
      x(t) &= \frac{1}{\sqrt{2\pi}}\int_{-\infty}^\infty X(\omega) e^{i\omega t}\,d\omega\\
      {} &= \frac{1}{\sqrt{2\pi}}\int_{-{2\pi B}}^{2\pi B} X(\omega) e^{i\omega t}\,d\omega\\
      {} &= \frac{1}{\sqrt{2\pi}}\int_{-{2\pi B}}^{2\pi B} \frac{\sqrt{2\pi}}{{4\pi B}}\sum_{k=-\infty}^\infty
      x\left(\frac{k}{2B}\right)e^{-i\pi k\omega/{2\pi B}} e^{i\omega t}\,d\omega\\
      {} &= \frac{1}{{4\pi B}}\int_{-{2\pi B}}^{2\pi B} \sum_{k=-\infty}^\infty x\left(\frac{k}{2B}\right)e^{-i\pi k\omega/{2\pi B}} e^{i\omega t}\,d\omega\\
      {} &= \frac{1}{{4\pi B}}\ \sum_{k=-\infty}^\infty x\left(\frac{k}{2B}\right)\int_{-{2\pi B}}^{2\pi B} e^{-i\pi k\omega/{2\pi B}} e^{i\omega t}\,d\omega
    \end{align*}
    If we evaluate the integral inside the summation we get:
    \[\int_{-{2\pi B}}^{2\pi B} e^{-i\pi k\omega/{2\pi B}} e^{i\omega t}\,d\omega = 2\sinc(\frac{{2\pi B} t}{\pi}-k)\]
    Which substituting back in gets us:
    \[ x(t) = \sum_{k=-\infty}^\infty x\left(\frac{k\pi}{{2\pi B}}\right)
    \sinc\left(\frac{{2\pi B} t}{\pi}-k\right) = \sum_{k=-\infty}^\infty x\left(\frac{k}{2B}\right)\sinc(2Bt-k) \]
  \end{proof}
\end{document}
