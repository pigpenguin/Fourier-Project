\documentclass [../article.tex]{subfiles}
\begin{document}
  \section{Signal Processing}
  Let $x(t)$ be some signal and $X(\omega)$ be it's
  spectrum, which is a map from a frequencey to it's
  strength in the signal.

  We can see that if you know $x(t)$ you can compute $X(\omega)$
  and if you know $X(\omega)$ you can compute $x(t)$.
  We will assume that your signal is band limited (no
  frequencies above and below a certain threshold) and
  nice enough to have a fourier series representation
  (sound is always this nice).

  Cool so the relation between $x(t)$ and $X(\omega)$ is:
  \[x(t) = \frac{1}{2\pi}\int_{-\infty}^\infty
  X(\omega)e^{it\omega}\,d\omega\]
  However as we are band limited we know that if
  $|\omega| > \frac{B}{2\pi}$ for some $B$ that
  $X(\omega) = 0$ thus our integral becomes
  \[x(t) = \frac{1}{2\pi}\int_{-2\pi B}^{2\pi B}
  X(\omega)e^{it\omega}\,d\omega\]
  Nearly done. All that is left is to sample our $x(t)$ at
  regular intervals, for instance $\frac{n}{2B}$ where $n \in
  \mathbb{Z}$.
  \[x\left(\frac{n}{2B}\right) =
  \int_{-2\pi B}^{2\pi B}X(\omega)e^{\frac{itn}{2B}}\,d\omega\]
  This, however, is just the $n^{\text{th}}$ fourier coeffecient
  of $X(\omega)$ so we have:
  \[x\left(\frac{n}{2B}\right) = \hat{X}(n)\]
  Thus if we sample our signal at regular intervals of
  $n/2B$ we will get our fourier series for $X(n)$ which
  uniquely determines $X(n)$ which uniquely determines $x(t)$.

\end{document}
